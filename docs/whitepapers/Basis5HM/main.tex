\documentclass[11pt]{article}
\usepackage[margin=1in]{geometry}
\usepackage{graphicx}
\usepackage{amsmath,amssymb,amsthm}
\usepackage{hyperref}
\title{LEE Toroidal Manifolds and Phase-Tensor Formalism (Basis5HM)}
\author{William Alexander Patterson IV}
\date{\today}

\newtheorem{definition}{Definition}
\newtheorem{lemma}{Lemma}
\newtheorem{theorem}{Theorem}

\begin{document}
	\maketitle
	
	\begin{abstract}
		We formalize LEE’s toroidal manifold dynamics and develop a phase-tensor intermediate representation (IR) connecting material implication to rotation operators. We state conditions under which counterfactual manifolds admit low-dimensional embeddings and identify limits (Basis5 Taxonomy I/II/III).
	\end{abstract}
	
	\section{Preliminaries}
	Operators (alive, jam, mem, vac), counterfactual entailment, notation.
	
	\section{Phase-Tensor IR}
	\begin{definition}[Phase-Tensor]
		Formal definition; composition; covariant properties.
	\end{definition}
	Compile rules; relation to implication table; conserved quantities.
	
	\section{Toroidal Manifolds}
	Embedding lemmas; rotation fields; stability regions; link to StressIndex.
	
	\section{Complexity and Exhaustibility}
	Place results in Class I/II/III; proof obligations.
	
	\section{Discussion and Outlook}
	\textit{Not inspired by neuroscience; later observed parallels.}
	
	\bibliographystyle{plain}
	\bibliography{../refs/LEE_refs}
\end{document}
