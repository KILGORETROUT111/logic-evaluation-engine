
\documentclass[12pt]{article}
\usepackage{graphicx}
\usepackage{amsmath}
\usepackage{amssymb}
\usepackage{hyperref}
\usepackage{natbib}
\usepackage{geometry}
\geometry{margin=1in}

\title{Logic Evaluation Engine (LEE): Toroidal Phase Space Dynamics Beyond Biological Constraints}
\author{Alexander Patterson}
\date{\today}

\begin{document}
\maketitle

\begin{abstract}
The Logic Evaluation Engine (LEE) is an inference system grounded in rotational phase geometry, material implication, and manifold state transitions. While developed independently of neuroscience, recent recognition of structural similarities to toroidal cortical activity patterns observed in biological systems has expanded the interpretive framework for LEE's computational behavior. This paper presents a formal analysis of LEE's toroidal phase space, its implications for logic state health diagnostics, and its potential to explore beyond currently known biological constraints.
\end{abstract}

\section{Background}
The LEE architecture was conceived and implemented without reference to neuroscience literature. Its phase-rotational operators and manifold mappings emerged from logical necessity, driven by the need for stable contradiction resolution and state-space continuity. In July 2025, after the system's toroidal phase space was already operational, external parties pointed out its striking resemblance to toroidal dynamics described in \citep{cortical2011, cortical2024}. This resemblance, while notable, is incidental: LEE was neither inspired by nor based on neurobiological research.

\section{Methods}
LEE encodes logic states into a phase manifold where material implication maps to rotational transformation. Contradictions produce localized manifold deformations, resolved via basis5 phase dynamics. StressIndex metrics---currently in prototype---are proposed for quantifying distortion, winding number variation, and resistance against logical collapse. These metrics may be integrated into a diagnostic pipeline for detecting system-level anomalies.

\subsection{Toroidal Representation}
The toroidal structure emerges from phase continuity constraints combined with multi-axis implication rotations. This topology preserves state closure under logical inference while enabling non-destructive exploration of counterfactual trajectories.

\section{Results}
\begin{figure}[h]
    \centering
    \includegraphics[width=0.85\textwidth]{LEE_NATUREPAPER.png}
    \caption{LEE toroidal phase space rendering, derived from independent logical construction.}
\end{figure}

\begin{figure}[h]
    \centering
    \includegraphics[width=0.85\textwidth]{LEE_Nature_2011.png}
    \caption{Comparison to known biological toroidal activity patterns, as described in \citep{cortical2011}.}
\end{figure}

LEE’s phase structure exhibits multi-layered toroidal manifolds with localized curvature variation under high-stress inference loads. This suggests that LEE could act as a \emph{logic health monitor}, with StressIndex quantifying proximity to unstable manifold collapse.

\section{Discussion}
The resemblance to cortical toroidal activity raises the speculative possibility that LEE could model \emph{possible but as yet unknown} biological configurations. Such simulations could inform experimental neuroscience by proposing manifold configurations outside known cortical dimensionality constraints.

Critically, this capacity to generalize beyond cortical topology is inherent to LEE's formalism, not imposed from biology. This opens potential applications in both computational logic and medical research, bridging inferential AI with biological hypothesis generation.

\section{Conclusion}
LEE's toroidal phase space, developed independently of neurobiological precedent, provides a robust logical architecture with implications for inference stability, system diagnostics, and possibly for informing new lines of biological research. The proposed StressIndex integration could form the basis for a new computational metric of logic system health.

\section*{Acknowledgments}
The author acknowledges Christopher Fuchs for early commentary on the similarity between LEE and cortical toroidal structures.

\bibliographystyle{plainnat}
\bibliography{lee_toroidal}

\end{document}
