
\documentclass[12pt]{article}
\usepackage{amsmath, amssymb, amsthm}
\usepackage{graphicx}
\usepackage{hyperref}
\usepackage{geometry}
\geometry{margin=1in}

\title{Toroidal Phase Space and Beyond-Biological Manifold Dynamics in the Logic Evaluation Engine (LEE)}
\author{William Alexander Patterson}
\date{\today}

\begin{document}

\maketitle

\begin{abstract}
The Logic Evaluation Engine (LEE) exhibits a toroidal phase space structure emergent from its treatment of material implication as a phase rotation operator. 
This white paper details the independently-discovered toroidal topology, its convergence with recent neuroscience literature, and its extension beyond biological constraints. 
We formalize manifold distortion metrics to evaluate logical state "health" and integrate the StressIndex framework to quantify resilience and instability within logical inference flows. 
Finally, we explore speculative but plausible medical applications where LEE could model potential biological configurations not yet observed in nature, providing a computational lens for discovery.
\end{abstract}

\section{Background}
LEE was not inspired by neuroscience research, but rather developed from first principles in computational logic. 
In mid-2025, external observers noted its strong visual and structural similarity to toroidal low-dimensional embeddings of high-dimensional brain activity (Nature, 2011; Nature, 2024). 
This observation came after the explicit development of a toroidal phase space within LEE's phase geometry module. 
Given LEE's independence from these inspirations, the similarities are convergent, arising from shared geometric constraints in dynamical systems.

\section{Mathematical Framework}
LEE models logical propositions as state vectors in a high-dimensional phase space. 
Material implication, $P \rightarrow Q$, is operationalized as a rotation in a complex-valued tensor space, mapping onto a toroidal embedding when projected into low dimensions.

Let $\mathbf{x}(t) \in \mathbb{C}^n$ be the logical state vector at time $t$. 
The core update rule for material implication is:
\begin{equation}
\mathbf{x}(t + \Delta t) = R_{\theta} \mathbf{x}(t),
\end{equation}
where $R_{\theta}$ is a unitary rotation matrix with $\theta$ derived from the logical truth table's phase geometry.

The toroidal structure arises when constraining:
\begin{align}
\|\mathbf{x}\| &= r_1, \\
\|\Pi_{\perp} \mathbf{x}\| &= r_2,
\end{align}
with $r_1$ and $r_2$ representing the major and minor radii, respectively. The dynamics preserve these norms under ideal inference conditions.

\section{Manifold Distortion Metrics}
To quantify the ``health'' of LEE's logical state space, we define a distortion metric $D$ comparing the instantaneous manifold embedding to the idealized torus:
\begin{equation}
D(t) = \frac{1}{N} \sum_{i=1}^N \left| \|\mathbf{x}_i(t)\| - r_1 \right| + \left| \|\Pi_{\perp} \mathbf{x}_i(t)\| - r_2 \right|.
\end{equation}
High $D(t)$ indicates deviation from the invariant manifold, analogous to neural misfiring or logical instability.

\section{StressIndex Integration}
The StressIndex $S$ is defined as a time-weighted integral over distortion:
\begin{equation}
S = \frac{1}{T} \int_{0}^{T} w(t) D(t) \, dt,
\end{equation}
where $w(t)$ is a weighting function emphasizing recent deviations. 
This formalism allows LEE to diagnose not only static distortion but also temporal instability patterns, enabling intervention or reconfiguration of logical pathways before critical failures occur.

\section{Beyond-Biological Constraints}
While analogous to structures in biological systems (LEE was developed independently of research in biological systems), LEE's dimensionality is not limited by cortical topology or evolutionary constraints. 
Its manifold embedding can extend to $n \gg 3$ while retaining toroidal projections in select subspaces. 
We hypothesize that such embeddings could model potential biological configurations beyond currently known limits, offering predictive scenarios for neuroinformatics and synthetic cognition research.

\section{Discussion}
The independent emergence of a toroidal manifold in LEE suggests that certain geometric attractors are universal to information-processing systems, regardless of substrate. 
By formalizing manifold distortion metrics and integrating them with the StressIndex, we gain a rigorous diagnostic for logical system stability.

Speculatively, LEE could provide a simulation platform for exploring hypothetical biological manifolds. 
For example, distortions that lead to pathological logical inference in LEE might correspond to undetected or rare neurological conditions. 
Conversely, stable configurations discovered in silico might inspire new therapeutic targets or AI-augmented neuroprosthetics.

\section{Conclusion}
LEE's toroidal phase space is both a practical computational construct and a bridge to interdisciplinary research. 
By combining strict logical formalism with geometric diagnostic tools, we open pathways to novel AI architectures and potentially transformative biomedical applications.

\bibliographystyle{unsrt}
\bibliography{lee_toroidal}
\end{document}
