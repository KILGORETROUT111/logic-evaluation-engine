
\documentclass[12pt]{article}
\usepackage[utf8]{inputenc}
\usepackage{amsmath, amssymb}
\usepackage{graphicx}
\usepackage{hyperref}
\usepackage[round]{natbib}
\usepackage{geometry}
\geometry{margin=1in}

\title{Logic Evaluation Engine (LEE) and Low-Dimensional Toroidal Manifolds:\\
Extending Computational Inference Beyond Biological Constraints}
\author{Alexander Patterson}
\date{\today}

\begin{document}
\maketitle

\begin{abstract}
The Logic Evaluation Engine (LEE) is a computational framework designed to perform high-fidelity logical inference through phase-based state geometry and counterfactual reasoning. Independently conceived without influence from neuroscience research, LEE’s architecture was later observed to share structural parallels with toroidal low-dimensional manifolds reported in neuroscience literature. This resemblance was pointed out post-development, most notably in relation to a 2011 \textit{Nature} study and its 2024 follow-up. The present whitepaper outlines the mathematical foundations of LEE, discusses the implications of these parallels, and proposes a forward-looking research agenda where LEE may inform medical and cognitive science by extending beyond known biological constraints.
\end{abstract}

\section{Background}
The LEE architecture originated from a formal logic and computational geometry perspective. Its phase state geometry emerged from the mapping of material implication to phase rotation operators, yielding a natural toroidal phase space. This was not inspired by brain topology research; rather, the connection was recognized only after peers noted similarities to neuroscience findings on low-dimensional toroidal manifolds in cortical dynamics.

In 2011, \citet{Nature_2024} reported that neural activity during certain cognitive tasks could be projected onto low-dimensional tori embedded within a high-dimensional state space. While LEE was not influenced by such research, the formal overlap suggests that similar mathematical constraints may govern diverse dynamical systems.

\section{Methods}
LEE's computational framework employs:
\begin{enumerate}
    \item \textbf{Phase Geometry:} Logical states are mapped to positions in a multi-dimensional phase manifold, where logical operations correspond to phase rotations.
    \item \textbf{Counterfactual Reasoning:} State transitions are modeled with counterfactual operators that permit evaluation of alternative histories.
    \item \textbf{StressIndex:} A metric for detecting manifold distortion, potentially applicable to diagnosing “logic health” in both computational and biological systems.
    \item \textbf{Global Memory Topology:} A structured memory addressing system supporting non-linear state recall.
\end{enumerate}

\section{Results and Observations}
Preliminary visualization of LEE’s phase state evolution reveals rotational trajectories and embeddings reminiscent of toroidal activity observed in biological systems. However, LEE’s dimensionality is not fixed to cortical topology, permitting exploration of logical state spaces beyond known neurobiological limits.

\section{Speculative Extensions}
While LEE’s immediate domain is computational inference, the possibility arises that its modeling of state dynamics could aid in hypothesizing previously unknown biological constraints. If validated, this could open a new class of medical and cognitive diagnostic tools.

\section{Discussion}
The resemblance between LEE and observed neural manifolds may be coincidental, but it may also point to a deeper universality in phase-driven systems. Further cross-disciplinary research could explore whether such manifolds reflect fundamental constraints in dynamical systems capable of sustained inference.

\section{Conclusion}
LEE was developed independently of neuroscience research, yet exhibits structural similarities to findings in cortical dynamics. These parallels suggest fertile ground for interdisciplinary exploration and application, particularly in diagnostics that integrate logical manifold analysis.

\bibliographystyle{plainnat}
\bibliography{references}

\end{document}
