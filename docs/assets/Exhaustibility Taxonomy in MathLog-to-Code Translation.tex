\section{Exhaustibility Taxonomy in MathLog-to-Code Translation}

A critical consideration in the Logic Evaluation Engine (LEE) Basis5 framework is the extent to which 
mathematical-logical (MathLog) structures can be translated into executable code. 
We introduce a taxonomy of \emph{exhaustibility}, distinguishing three classes of feasibility. 
This provides a structured lens for evaluating which aspects of LEE are implementable with 
current resources, which await technological translation, and which are non-implementable in principle.

\subsection{Exhaustible-A: Resource-Constrained Implementable}
\textbf{Definition:} Logical structures that are computable in principle and 
possess feasible algorithmic implementations, but are blocked by resource constraints 
(e.g., scale, memory, time, compute). 

\textbf{Analog in complexity theory:} Problems in P, NP, PSPACE.  
\textbf{Examples in LEE:} Multi-patient counterfactual inference where the branching factor 
is computationally tractable but requires high-performance clusters to execute.  

\textbf{Proof obligation:} Demonstrate algorithmic feasibility under standard computational models, 
with explicit resource bounds. Approximations or reductions are admissible.  

\subsection{Exhaustible-B: Translator-Lagged Implementable}
\textbf{Definition:} Logical structures that are implementable in principle but lack 
media-translators, compilers, or architectures to bridge MathLog constructs into executable form.  

\textbf{Analog in computer science:} Unexploited Curry--Howard correspondences, 
quantum computational substrates, or unimplemented category-theoretic programming paradigms.  

\textbf{Examples in LEE:} Counterfactual manifolds requiring phase-geometric compilers 
for Alive--Jam--Mem--Vac rotations; logically finite but outside the expressive power 
of existing programming languages.  

\textbf{Proof obligation:} Constructive proofs of implementability in theory, 
plus specification of missing translators or substrates.  

\subsection{Non-Exhaustible: Theoretically Closed}
\textbf{Definition:} Logical structures that cannot be exhaustively translated 
to code in any finite model, regardless of resources or translators.  

\textbf{Analog in theory:} Gödel incompleteness, Turing undecidability, 
non-constructive objects in higher-order logics.  

\textbf{Examples in LEE:} Infinite regress in counterfactual entailment loops, 
where closure is well-defined mathematically but not representable 
in any Turing-equivalent machine.  

\textbf{Proof obligation:} Proofs of impossibility (reductions to undecidability, 
diagonalization, or constructive non-representability).  

\subsection{Evolutionary Path}
The taxonomy is not static. Resource-constrained cases may, with technological 
advancement, shift into Translator-Lagged and eventually to Implementable. 
Conversely, certain Translator-Lagged cases may collapse into Non-Exhaustible 
if shown to require inherently non-computable constructs.

\begin{figure}[h]
	\centering
	\includegraphics[width=0.8\linewidth]{figures/exhaustibility_triangle.png}
	\caption{Exhaustibility taxonomy in MathLog-to-Code translation: 
		Resource-Constrained (A), Translator-Lagged (B), and Non-Exhaustible. 
		Arrows indicate possible research trajectories and reclassification 
		over time.}
\end{figure}

\subsection{Implications for LEE}
By situating LEE’s operators (Alive, Jam, Mem, Vac) and its counterfactual 
rotations within this taxonomy, we clarify the research agenda:  
\begin{itemize}
	\item \textbf{Exhaustible-A:} Demonstrate feasibility on HPC platforms.  
	\item \textbf{Exhaustible-B:} Formalize specifications for missing translators.  
	\item \textbf{Non-Exhaustible:} Define theoretical boundaries where 
	Basis5 reaches beyond code into theoretical informatics.  
\end{itemize}

This taxonomy both grounds LEE in the computability tradition and 
extends it into a novel engineering framework for logic engines.
