
\documentclass[12pt]{article}
\usepackage{amsmath, amssymb, amsthm}
\usepackage{graphicx}
\usepackage{hyperref}
\usepackage{geometry}
\usepackage{natbib}
\usepackage{float}
\usepackage{caption}
\usepackage{subcaption}
\usepackage{svg} % requires inkscape for direct svg include

\geometry{margin=1in}
\title{LEE: Toroidal Phase Space Dynamics Beyond Known Biological Constraints}
\author{William Alexander Patterson}
\date{\today}

\begin{document}
\maketitle

\begin{abstract}
LEE’s toroidal manifold emerged independently from neuroscience and follows from phase-rotation necessity in material implication. 
We formalize distortion metrics and the StressIndex and present current, real visuals emitted by the engine to substantiate claims.
\end{abstract}

\section{Framework Recap (Condensed)}
Material implication is realized as a rotation in a discrete phase alphabet $\{\textsf{ALIVE},\textsf{JAM},\textsf{MEM}\}$ with angles in $\{0^\circ,90^\circ,180^\circ,270^\circ\}$. 
Two independent cycles (evaluation and archival) form $\mathbb{T}^2$. Distortion under load is quantified and drives the StressIndex.

\section{Figures: Current LEE Outputs}
\begin{figure}[H]\centering
\includegraphics[width=0.92\linewidth]{figures_wp/lee_naturepaper.png}
\caption{LEE toroidal phase-space rendering derived from the logical manifold (independent of neuroscience).}
\label{fig:lee-toroidal}
\end{figure}

\begin{figure}[H]\centering
\includegraphics[width=0.92\linewidth]{figures_wp/lee_nature_2011.png}
\caption{Comparison panel historically associated with 2011 observations of toroidal activity; presented here as a visual analogy only.}
\label{fig:lee-nature-analogy}
\end{figure}

\begin{figure}[H]\centering
\includegraphics[width=0.9\linewidth]{}
\caption{Cumulative winding from a recent LEE run; StressIndex is derived from winding and phase occupancy (JAM ratio).}
\label{fig:stressindex-winding}
\end{figure}

\begin{figure}[H]\centering
\includesvg[width=0.92\linewidth]{figures_wp/run_svg.svg}
\caption{Direct SVG emission from a provenance-backed LEE run (timeline/graph). For arXiv, convert to PDF if preferred.}
\label{fig:lee-svg}
\end{figure}

\section{Manifold Distortion and StressIndex (Pointer)}
Let $D(t)$ be the deviation from ideal torus (major/minor radii drift and curvature concentration). 
Define the StressIndex $S=\frac{1}{T}\int_0^T w(t)D(t)\,dt$ with phase occupancy weighting; in absence of durations, use count-based JAM ratio. 
These figures demonstrate present capability; subsequent versions will include time-weighted variants when phase durations are logged.

\section{Notes for Reviewers}
All visuals shown are \emph{direct engine outputs}. They are not schematic sketches but artifacts generated by LEE’s evaluation pipeline.
SVGs are included to preserve vector fidelity; compile with \texttt{--shell-escape} or convert to PDF/PNG.

\bibliographystyle{plainnat}
\bibliography{lee_toroidal}
\end{document}
